\documentclass[paper=a4paper,titlepage]{jlreq}
\usepackage[top=2cm, bottom=3cm, left=2cm, right=2cm]{geometry}
\usepackage{enumitem}
\usepackage{empheq}
\usepackage{siunitx}
\usepackage{tikz}
\usetikzlibrary{math,calc,quotes,angles}
\usepackage{tikz-3dplot}
\usepackage{xcolor}
\usepackage{wrapfig}
\usepackage{ascmac}
\usepackage{docmute}
\pagestyle{empty}
\begin{document}
\underline{氏名\hspace{330pt}得点\hspace{70pt}/25\hspace{10pt}}
\begin{itembox}[l]{問題}
    Oを原点とするxy平面上に、正方形OABCがある。P(4,0)が辺AB上に、Q($2\sqrt{3}$\hspace{1pt},\hspace{1pt}2)が辺BC上にあるとき、次の各問いに答えなさい。
    \\(1)\hspace{3pt}$\mathrm{OQ}$、$\mathrm{\angle{QOP}}$を求めなさい。
    \\(2)正方形の1辺の長さ、および、Bの座標を求めなさい。
    \\(3)\hspace{3pt}OPを折り目として$\triangle{\mathrm{OAP}}$を折り返し、Aが移る点をA'とする。
    次に、BC上の点Rに対して、ORを折り目としてOCがOA'に重なるように折り返す。このとき、直線ACとORの交点をDとする。
    \\\hspace{15pt}($\mathrm{\hspace{.18em}i\hspace{.18em}}$)直線ORの式を求めなさい。
    \\\hspace{15pt}($\mathrm{\hspace{.08em}ii\hspace{.08em}}$)\hspace{3pt}$\mathrm{PD\perp OR}$を証明しなさい。
\end{itembox}
\end{document}